\documentclass[11pt]{article}

\usepackage{ods_etsinf}

\begin{document}

\phantom{x}

\vspace{1ex}

\section*{ANEXO}

\vspace{2ex}

OBJETIVOS DE DESARROLLO SOSTENIBLE

\vspace{4ex}

Grado de relaci\'on del trabajo con los Objetivos de Desarrollo Sostenible (ODS).

\vspace{2ex}

\begin{tabular}{|l|c|c|c|c|}\hline
\textbf{Objetivos de Desarrollo Sostenible} & \textbf{Alto} & \textbf{Medio} & \textbf{Bajo} & \textbf{No} \\
& & & & \textbf{procede} \\ \hline
ODS 1.  \textbf{Fin de la pobreza.}                            & & & & \textbf{X} \\ \hline
ODS 2.  \textbf{Hambre cero.}                                  & & & & \textbf{X} \\ \hline
ODS 3.  \textbf{Salud y bienestar.}                            & & \textbf{X} & & \\ \hline
ODS 4.  \textbf{Educaci\'on de calidad.}                       & \textbf{X} & & & \\ \hline
ODS 5.  \textbf{Igualdad de g\'enero.}                         & & & & \textbf{X} \\ \hline
ODS 6.  \textbf{Agua limpia y saneamiento.}                    & & & & \textbf{X} \\ \hline
ODS 7.  \textbf{Energ\'{\i}a asequible y no contaminante.}     & & & & \textbf{X} \\ \hline
ODS 8.  \textbf{Trabajo decente y crecimiento econ\'omico.}    & & \textbf{X} & & \\ \hline
ODS 9.  \textbf{Industria, innovaci\'on e infraestructuras.}   & & \textbf{X} & & \\ \hline
ODS 10. \textbf{Reducci\'on de las desigualdades.}             & & \textbf{X} & & \\ \hline
ODS 11. \textbf{Ciudades y comunidades sostenibles.}           & & & & \textbf{X} \\ \hline
ODS 12. \textbf{Producci\'on y consumo responsables.}          & & & & \textbf{X} \\ \hline
ODS 13. \textbf{Acci\'on por el clima.}                        & & & & \textbf{X} \\ \hline
ODS 14. \textbf{Vida submarina.}                               & & & & \textbf{X} \\ \hline
ODS 15. \textbf{Vida de ecosistemas terrestres.}               & & & & \textbf{X} \\ \hline
ODS 16. \textbf{Paz, justicia e instituciones s\'olidas.}      & & & \textbf{X} & \\ \hline
ODS 17. \textbf{Alianzas para lograr objetivos.}               & & & \textbf{X} & \\ \hline
\end{tabular}

\newpage

\vspace*{3ex}

\textbf{Reflexi\'on sobre la relaci\'on del TFG/TFM con los ODS y con el/los ODS m\'as relacionados.}

\vspace{1ex}

%\textit{Empieza a escribir aquí tu reflexión. Aproximadamente entre 500 y 1500 palabras. Se espera que puedas relacionar tu TFG con alguno (o varios) ODS. En el caso de no poder relacionarse, deberás explicar porqué no es posible hacerlo y/o identificar aquellos ODS que en menor grado podrían tener alguna relación con el trabajo. Recuerda Borrar toda esta frase}

El TFG presentado tiene una relación significativa con varios de los Objetivos de Desarrollo Sostenible (ODS) de la Agenda 2030 propuestos por la Organización de las Naciones Unidas (ONU). A continuación, se muestra una reflexión sobre cómo este trabajo contribuye a algunos de estos objetivos.

\vspace{1ex}

La relación más directa y significativa de este TFG es con el \textbf{ODS 4 Educación de calidad}, que se enfoca en garantizar una educación inclusiva, equitativa y de calidad. El desarrollo de sistemas de traducción automática capaces de generar traducciones de una calidad similar a las traducciones manuales realizadas por traductores expertos, elimina las barreras lingüísticas, facilitando el acceso a información y recursos educativos independientemente de su idioma nativo. 

\vspace{1ex}

La relación de este TFG con el \textbf{ODS 3 Salud y bienestar} se manifiesta principalmente a través del impacto que las tecnologías avanzadas de traducción automática pueden tener en el ámbito de la salud y el bienestar. La traducción precisa y eficiente de información médica es crucial para asegurar que profesionales de la salud, investigadores y pacientes tengan acceso a información vital sin barreras lingüísticas. Al mejorar las herramientas de traducción automática, este TFG contribuye a la accesibilidad de estudios clínicos, guías médicas, y material educativo sobre salud en múltiples idiomas, permitiendo que la información crítica esté al alcance de todos.

\vspace{1ex}

El \textbf{ODS 8 Trabajo decente y crecimiento económico} busca promover el crecimiento económico sostenido, inclusivo y sostenible, el empleo pleno y productivo, y el trabajo decente para todos. Las mejoras en las tecnologías de traducción automática tienen un impacto positivo en la productividad y eficiencia laboral. En un mundo cada vez más globalizado, la capacidad de comunicarse efectivamente en múltiples idiomas es crucial para las empresas y organizaciones. La implementación de sistemas avanzados de traducción puede mejorar la comunicación interna y externa, reducir costos de traducción, y permitir una colaboración más fluida y efectiva entre equipos multiculturales.

\vspace{1ex}

El \textbf{ODS 9 Industria, innovación e infraestructura} promueve la construcción de infraestructuras resilientes, la industrialización inclusiva y sostenible, y el fomento de la innovación. Este TFG, al explorar y mejorar las técnicas de traducción automática mediante el uso de LLMs y enfoques de ajuste eficiente de parámetros, se alinea claramente con la promoción de la innovación tecnológica. La investigación y el desarrollo en el campo de la traducción automática representan un avance significativo en la creación de infraestructuras tecnológicas que pueden ser aplicadas en diversos sectores industriales, desde la educación hasta la comunicación y el comercio internacional.

\vspace{1ex}

En cuanto al \textbf{ODS 10 Reducción de las desigualdades}, que se centra en reducir las desigualdades entre los países y dentro de ellos. La traducción automática desempeña un papel crucial en la reducción de las desigualdades lingüísticas. Al facilitar la comunicación y el acceso a la información en diferentes idiomas, estas tecnologías pueden ayudar a que extranjeros que no conocen el idioma principal del país en el que residen pueden integrarse en la sociedad. Además, los sistemas de traducción automática, especialmente para idiomas de bajos recursos, pueden contribuir a que comunidades marginadas puedan integrarse y relacionarse internacionalmente, fomentando así su participación en el comercio global.

\newpage

\vspace*{1ex}

La relación de este TFG con el \textbf{ODS 16 Paz, justicia e instituciones sólidas} se refleja en cómo las tecnologías de traducción automática pueden facilitar la comunicación y la cooperación entre diferentes culturas y lenguas, promoviendo así la paz y la comprensión mutua. En el contexto de instituciones de justicia y gobernanza, la capacidad de traducir documentos legales y administrativos de manera precisa y eficiente es crucial para garantizar que todas las partes involucradas comprendan plenamente sus derechos y obligaciones, independientemente de su lengua materna. Además, la accesibilidad a la información pública en varios idiomas fortalece la transparencia y la responsabilidad de las instituciones gubernamentales, lo que es fundamental para construir confianza y legitimidad.

En el caso del \textbf{ODS 17 Alianzas para lograr objetivos}, este se centra en revitalizar la alianza global para el desarrollo sostenible. Este TFG contribuye a la exploración y el desarrollo de herramientas de traducción automática, lo que es fundamental para una colaboración internacional efectiva. Los sistemas de traducción pueden fomentar la cooperación entre gobiernos, empresas privadas e individuos de diferentes países o regiones. Esto resulta especialmente importante en un mundo globalizado donde la cooperación internacional es clave para abordar desafíos globales como el cambio climático, la salud pública y la desigualdad económica.

\end{document}
